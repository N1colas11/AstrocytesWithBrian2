\documentclass{article}
\usepackage{amsmath}
%\usepackage[utf8]{inputenc}
%\usepackage[table]{xcolor}
%-----------------------------------------------
\title{Volumetric effects in astrocyte branching}
\author{ }
\date{January 2020}
%-----------------------------------------------
\begin{document}

\maketitle
\newcommand{\trm}{\textrm}

\section{Introduction}

\section{Parameters}
%Table~\ref{tab:params} lists the parameters of the problem. 
\begin{center}
% args: row to start, color for odd rows, color for even rows
%{\rowcolors{2}{green!80!yellow!50}{green!70!yellow!40}
\begin{tabular}{|l|l|l|l|} 
\hline
Symbol & Definition & Unit & Value \\
\hline
$C_e$ & Extracellular calcium concentration & $1800$ & ${\mu\trm{Mole}}$ \\
$C_{rest}$ & Resting concentration & 0.073 & $\mu\trm{Molar}$ \\
$p_{Ca}$ & Calcium permeability & $4.46\times 10^{-13}$ & $\trm{cm s}^{-1}$  \\
$V_T$ & Thermal potential& $26$& $\trm{mV}$ \\
$V_{rest}$& Resting potential & $-80$ & $\trm{mV}$ \\
$F$ & Faraday Constant &$96485.3329$ &$\trm{A s Mole}^{-1}$   \\
$k_B$ & Boltzman constant&$1.38064852\times 10^{-23}$ & $\trm{m}^2 \trm{kg s}^{-2} \trm{K}^{-1}$ \\
$N_A$ & Avogradro number& $6.02214076 \times 10^{23}$& $\trm{Mole}^{-1}$ \\
$e$ & Electron charge& $1.602176634\times 10{-19}$& $\trm{A s}$ \\
$C_m$ & Membrane capacity& 1 &$\mu\trm{F}^2 \trm{m}^{-2}$  \\
$R_{gas}$ & Gas Constant& 8.31 & $\trm{J K}^{-1} \trm{Mole}^{-1}$\\
$D_C$ & Calcium Diffusion Constant& $5.3 \times 10^{-6}$ & $\trm{cm}^2/\trm{s}$ \\   % <<< FIRST ERROR
$R$ & Compartment radius & 3 & $\mu\trm{m}$  \\
$r$ & ER radius & 2 & $\mu\trm{m}$  \\
$L$ & Compartment length& 8 & $\mu\trm{m}$  \\
$A$ & Current& 1 & $\trm{volt}^{-1}$ \\
$C$ & Initial value for cytosolic calcium & $1.1\times 10^{-4}$ & $\trm{Mole m}^{-3}$  \\
\hline
$d_1$ &   &     & \\
$d_2$ &   &     & \\
$d_3$ &   &     & \\
$d_5$ &   &     & \\
$d_1$ &   &     & \\
$d_1$ &   &     & \\
\hline
$d_3$ &   &     & \\
$d_5$ &   &     & \\
$d_1$ &   &     & \\
$d_1$ &   &     & \\
\hline
$d_3$ &   &     & \\
$d_5$ &   &     & \\
$d_1$ &   &     & \\
$d_1$ &   &     & \\
\hline
$d_3$ &   &     & \\
$d_5$ &   &     & \\
$d_1$ &   &     & \\
$d_1$ &   &     & \\
\hline
$d_3$ &   &     & \\
$d_5$ &   &     & \\
$d_1$ &   &     & \\
$d_1$ &   &     & \\
\hline
$d_3$ &   &     & \\
$d_5$ &   &     & \\
$d_1$ &   &     & \\
$d_1$ &   &     & \\
\hline
$d_3$ &   &     & \\
$d_5$ &   &     & \\
$d_1$ &   &     & \\
$d_1$ &   &     & \\
\hline
$d_3$ &   &     & \\
$d_5$ &   &     & \\
$d_1$ &   &     & \\
$d_1$ &   &     & \\
\hline
$d_3$ &   &     & \\
$d_5$ &   &     & \\
$d_1$ &   &     & \\
$d_1$ &   &     & \\
\hline
$d_3$ &   &     & \\
$d_5$ &   &     & \\
$d_1$ &   &     & \\
$Gordon$ &   &     & \\
\hline
\hline
\end{tabular} \\
Table~\cite{tab_params}. Model parameters.
\end{center}
%}

\begin{verbatim}
# Constants found in currents
# IP3 production
O_delta = 0.6*umolar/second  # Maximal rate of IP_3 production by PLCdelta
k_delta = 1.5* umolar        # Inhibition constant of PLC_delta by IP_3
K_delta = 0.1*umolar         # Ca^2+ affinity of PLCdelta   <<<<<<<<<<<<<<<<<

# IP3 degradation
K_D  = 0.7*umolar            # Ca^2+ affinity of IP3-3K
K_3  = 1.0*umolar            # IP_3 affinity of IP_3-3K
O_3K = 4.5*umolar/second     # Maximal rate of IP_3 degradation by IP_3-3K

# IP5 degradation
o_5P = 0.05/second           # Maximal rate of IP_3 degradation by IP-5P
K_5P = 10*umolar

# IP3 delta Production
o_delta = 0.6*umolar/second  # Maximal rate of IP_3 production by PLCdelta
k_delta = 1.5* umolar    # Inhibition constant of PLC_delta by IP_3
K_delta = 0.1*umolar         # Ca^2+ affinity of PLCdelta   <<<<<<<<<<<<<<<<<

# Volume of an average soma
Lambda = 2100*umeter**3
# Not sure about this
rho_A = 0.18                 # ER-to-cytoplasm volume ratio ?
rho = rho_A / (1.+rho_A)     # ER-to-cytoplasm volume ratio ?

# Multiply value of amplitudes by cytosolic volume (changed rho_A to rho)
o_delta = 0.6  * umolar * Lambda * (1-rho) / second
o_3K    = 4.5  * umolar * Lambda * (1-rho) / second
o_5P    = 0.05 * umolar * Lambda * (1-rho) / second

Omega_L = 0.1/second         # Maximal rate of Ca^2+ leak from the ER
Omega_2 = 0.2/umolar/second      # IP_3R binding rate for Ca^2+ inhibition
Omega_u = 0.2/umolar/second      # uptake/dissassociation constant

# --- IP_3R kinectics
d_1 = 0.13*umolar            # IP_3 binding affinity
d_2 = 1.05*umolar            # Ca^2+ inactivation dissociation constant
O_2 = 0.2/umolar/second      # IP_3R binding rate for Ca^2+ inhibition
d_3 = 0.9434*umolar          # IP_3 dissociation constant
d_5 = 0.08*umolar            # Ca^2+ activation dissociation constant

p_open = 1
P_r     = 1*umolar/second
P_CE    = 1*umolar/second

eta_p = 1
d_ER  = 1 * umeter

# Another complex calculation that depends on solution to a cubic equation
# eqs. (73)-(76) in De Pitta's notes. MUST IMPLEMNENT. 
# Later, evaluate the terms and figure out what is negligeable and what is not. 
dv_ER = 0 * mvolt

K_P = 0.05 * umolar          # Ca2+ affinity of SERCAs

#o_3K
#p_open
#Pr
#dv_ER
#d_ER
#N_A
#Omega_u
#eta_p

'''
ASTROCTE PARAMETERS
### Astrocyte parameters
# ---  Calcium fluxes
# Value provided by MDP
O_P = 1.0*umolar/second      # Maximal Ca^2+ uptake rate by SERCAs  (0.9 in MDP, 1.0 in Evan)
# Value must equal v_er in Evan model
#O_P = 4.4*umolar/second      # Maximal Ca^2+ uptake rate by SERCAs  (0.9 in MDP, 1.0 in Evan)

K_P = 0.05 * umolar          # Ca2+ affinity of SERCAs
C_T = 2*umolar               # Total cell free Ca^2+ content
rho_A = 0.18                 # ER-to-cytoplasm volume ratio
rho = rho_A / (1.+rho_A)     # ER-to-cytoplasm volume ratio
Omega_C = 6/second           # Maximal rate of Ca^2+ release by IP_3Rs
Omega_L = 0.1/second         # Maximal rate of Ca^2+ leak from the ER
# --- IP_3R kinectics
d_1 = 0.13*umolar            # IP_3 binding affinity
d_2 = 1.05*umolar            # Ca^2+ inactivation dissociation constant
O_2 = 0.2/umolar/second      # IP_3R binding rate for Ca^2+ inhibition
d_3 = 0.9434*umolar          # IP_3 dissociation constant
d_5 = 0.08*umolar            # Ca^2+ activation dissociation constant
# --- IP_3 production
O_delta = 0.6*umolar/second  # Maximal rate of IP_3 production by PLCdelta
kappa_delta = 1.5* umolar    # Inhibition constant of PLC_delta by IP_3
K_delta = 0.1*umolar         # Ca^2+ affinity of PLCdelta   <<<<<<<<<<<<<<<<<
# --- IP_3 degradation
O_5P = 0.05/second       # Maximal rate of IP_3 degradation by IP-5P
K_5P = 10*umolar
K_D = 0.7*umolar             # Ca^2+ affinity of IP3-3K
K_3K = 1.0*umolar            # IP_3 affinity of IP_3-3K
O_3K = 4.5*umolar/second     # Maximal rate of IP_3 degradation by IP_3-3K
# --- IP_3 diffusion
#F = 0.09*umolar/second    # Maximal exogenous IP3 flow
I_Theta = 0.3*umolar         # Threshold gradient for IP_3 diffusion
omega_I = 0.05*umolar        # Scaling factor of diffusion
'''

################################################################################
# Additional and modified parameters
################################################################################
# Volume of an average soma
Lambda = 2100*umeter**3

# Multiply value of amplitudes by cytosolic volume (changed rho_A to rho)
O_delta = 0.6  * umolar * Lambda * (1-rho) / second
O_3K    = 4.5  * umolar * Lambda * (1-rho) / second
O_5P    = 0.05 * umolar * Lambda * (1-rho) / second
F       = 0.1 / second
D       = 0.05 / second  # setting D to zero has no effect. Why?

print("**** Omega_L= ", Omega_L)
print("**** O_delta= ", O_delta) # 0.6
print("**** K_delta= ", K_delta) # 0.6

#----------------------------------------------------------------------
astro_eqs = '''
s = R + r          : meter
r                  : meter
dR2 = R**2 - r**2  : meter**2
DR2                : meter**2
L                  : meter
R                  : meter  # Outer radius
coupling_C         : mole/second/meter**3

####  CALCIUM

# has large impact on diffusion. Creates instability
coupling_electro   : mole / second / meter**3
#coupling_electro = 1  : 1 

# Tot_C is the total calcium (computed in synapses2)
Tot_C : mole/meter**3

V = Vrest + (F*s/Cm) * (Tot_C-Crest) : volt
VVT = -2.*V/V_T : 1  # value of 0, which leads to a zero denominator

# Has minimal impact on Calcium
electro_diffusion = -P_Ca * V / (R * V_T) * (Ce*exp(-2*V/V_T) - Tot_C) / (exp(-2*V/V_T) - 1) : mole / second / meter**3

A1 : meter**4 / mole 
B1 : meter 
C1 : mole / meter**2 

CC0 : mole / meter**3
V0 : volt
nb_connections : 1  # number of synaptic connections

# Calcium diffusion
# Each term will handle half a compartment 

dC/dt = 1*coupling_C + 1.*coupling_electro + 0.*electro_diffusion + Jr + J1 - Jp  : mole / meter**3

####  ENDOPLASMIC RETICULUM
Tot_CE                 : mole/meter**3
coupling_CE            : mole/second/meter**3
dCE/dt = 1*coupling_CE + Jp/(rho*Lambda) - (Jr + J1)  : mole/meter**3

####  IP3
Tot_I                  : mole/meter**3
coupling_I             : mole/second/meter**3
dI/dt = (Jbeta + Jdelta - J3K - J5P) / (Lambda*(1-rho)) + 1*coupling_CE   : mole/meter**3

### Open Channels
dh/dt = OmegaH * (hinf - h) : 1


################3
###  CURRENTS

Jr     = (2*r/dR2) * P_r * p_open * (CE-C) : mole/meter**3/second  # p_open, Pr
J1     = (4*P_CE/r*VVT) * dv_ER * (C*exp(-2.*dv_ER/VVT) - CE) / (exp(-2.*dv_ER/VVT)-1.) : mole/meter**3/second  # dv_ER
Jp     = (2*r*d_ER)/(N_A*dR2) * Omega_u * eta_p * C**2 / (C**2 + K_P**2) : mole/meter**3/second # d_ER, N_A, Omega_u, eta_p, K_p
minf   =  I / (I + d_1)  : 1 # d_1
ninf   = C / (C + d_5) : 1 # d_5
hinf   =  d_2 * (I + d_1) / (d_2*(I +d_1) + (I+d_3)*C) : 1 # d_2, d_1, d_3
OmegaH = (Omega_2*(I+d_1) + O_2*(I+d_3)*C) / (I + d_3) : Hz # Omega_2, O_2, d_1, d_3



#IP3 dynamics
Jbeta  = 0*mole/meter**3/second  : mole/meter**3/second
Jdelta = o_delta * (k_delta/(I+k_delta)) * (C**2/(C**2+K_delta**2)) : mole/second  # not sure about units, o_delta, k_delta, K_delta
J5P    = o_5P * (I/(I+K_5P)) : mole/second # o_5P, K_5P
J3K    = o_3K * (C**4/(C**4+K_D**4)) * (I/(I+K_3)) : mole/second # o_3K, K_D, K_3

'''

N_astro = 3 # Total number of astrocytes1 in the network (each compartment broken into two
N_astro = 2*N_astro # Total number of astrocytes1 in the network (each compartment broken into two
astrocytes1 = NeuronGroup(N_astro, astro_eqs, method='euler', name='ng1', order=1)

# Initital Conditions
astrocytes1.L = 8 *  umeter  # length of a compartment
astrocytes1.R = 3 * umeter
astrocytes1.r = 2 * umeter
# Calcium in 1/2 compartment should be the value for the full compartment since the 1/2 compartment
# Each 1/2 compartment only considers the fork at one end. 
# is an artifact of the numerical scheme.
astrocytes1.C = 1.1e-4  * mole / meter**3
astrocytes1.I = 1.1e-4  * mole / meter**3
astrocytes1.CE = 1.1e-4  * mole / meter**3
#astrocytes1.C = [1.1e-4, 1.5e-4, 1.6e-4]*2 * mole / meter**3

astro_mon = StateMonitor(astrocytes1, variables=['Tot_C', 'Tot_CE', 'Tot_I', 'coupling_C', 'coupling_electro', 'electro_diffusion', 'nb_connections', 'A1', 'B1', 'C1', 'CC0', 'V0'], record=True)
b_mon = StateMonitor(astrocytes1, variables=['CC0', 'dR2', 's', 'L'], record=True)

'''
One issue is Boundary Conditions. I have not done anything to implement those. 
Also, each synapse has its own value of V0. 
'''

#----------------------------------------------------------------------
# Diffusion between astrocytes1
synapse1_eqs = '''
# Assuming the above is correct, let us figure out units. 
# using C_pre:  C * F * s / Cm = V  ==> F / Cm = V / s / C = (V / s) * m^3 / mole

####  CYTOPLASM

# A1 stable: value does not change from iteration to iteration
A1_pre = (0.5 * F * s_post)/(Cm * V_T) * (dR2_post / L_post) : meter**4 / mole (summed)
B1_pre = (1. - (s_post * F * Tot_C_post)/(2. * Cm * V_T)) : meter (summed)
C1_pre = dR2_post * Tot_C_post / L_post : mole / meter**2 (summed)

# Only computed to trick Brian into computing C0_pre by summing the values for all synapses (which will 
#   all bequal  since it depends on A1, B1, C1, for which the summation has already been executed. 
# There must be an easier method without calculating C0 and V0 in the NeuronGroup
nb_connections_pre = 1 : 1 (summed)

# Notice I am dividing by the number of connections (3 if a single fork)
CC0_pre = ((-B1_pre + sqrt(B1_pre**2 + 4*C1_pre*A1_pre)) / (2.*A1_pre)) / nb_connections_pre : mole / meter**3 (summed)
V0_pre = (Vrest + (CC0_pre - Crest) * (F * s) / Cm) / nb_connections_pre   : volt  (summed)

# Investigate why this should be a minus sign
coupling_C_post = (4*D_C / L_post**2) * (Tot_C_pre - Tot_C_post) : mole/second/meter**3 (summed)

# MAKE SURE CC0 and V0 are computed before updating the coupling parameters. 
coupling_electro_pre = (4*D_C/L_post**2/V_T) * (CC0_post + Tot_C_post) * (V0_post - V_post) : mole/second/meter**3 (summed) 

####  ENDOPLASMIC RETICULUM
coupling_CE_post = (4*D_CE / L_post**2) * (Tot_CE_pre - Tot_CE_post) : mole/second/meter**3 (summed)

####  IP3
coupling_I_post = (4*D_I / L_post**2) * (Tot_I_pre - Tot_I_post) : mole/second/meter**3 (summed)
'''
#----------------------------------------------------------------------

# TEMPORARY
synapse2_eqs = '''
    # C is updated in each compartment of the astrocyte and combined here
	Tot_C_syn   = C_pre  + C_post    : mole / meter**3
	Tot_C_pre   = Tot_C_syn       : mole / meter**3 (summed)

	# update Calcium in compartments
	# This works because synapses are bidirectional i --> j and j --> i

####  ENDOPLASMIC RETICULUM
	Tot_CE_pre  = Tot_CE_syn      : mole / meter**3 (summed)
	Tot_CE_syn  = CE_pre + CE_post   : mole / meter**3

####  IP3
	Tot_I_syn = I_pre + I_post : mole / meter**3
	Tot_I_pre = Tot_I_syn      : mole / meter**3 (summed)
'''

synapses1 = Synapses(astrocytes1, astrocytes1, model=synapse1_eqs, method='euler', order=0, name='sg1')
synapses2 = Synapses(astrocytes1, astrocytes1, model=synapse2_eqs, method='euler', order=5, name='sg2')

# Connections count from 0
#  ---*---.---*----
#         |___*____

# Single fork (N_astro = 6)
# Six Compartments: 0,1,...,4,5
for i in range(N_astro):
    synapses1.connect(i=i,j=i)
pairs = [(3,1),(3,2),(1,2)]
for pair in pairs:
	synapses1.connect(i=pair[0], j=pair[1])
	synapses1.connect(i=pair[1], j=pair[0])

pairs = [(0,1),(2,4),(3,5)]
for pair in pairs:
	synapses2.connect(i=pair[0], j=pair[1])
	synapses2.connect(i=pair[1], j=pair[0])

s = synapses1.summed_updaters
s['nb_connections_pre'].order = 0
s['A1_pre'].order = 2
s['B1_pre'].order = 2
s['C1_pre'].order = 2
s['CC0_pre'].order = 4
s['V0_pre'].order = 5
s['coupling_C_post'].order = 6
s['coupling_electro_pre'].order = 7
\end{verbatim}


\section{Units of various terms}
Let us consider the different constants and expressions that appear in the equations that govern $I$, $C$, $Ce$, and $h$. 
We start with the Reaction current for the calcium equations. 
\newpage
\def\dvER{\Delta{v}_{ER}}
\def\dER{d_{ER}}
\def\dRtwo{\delta R^2}
\def\VVT{{V_T}}
\def\meter{\text{meter}}
\def\Hz{\text{Hz}}
\def\molar{\text{molar}}
\def\mole{\text{mole}}
\begin{align}
J_r &= \frac{2 r P_r p_{open}}{\dRtwo}   (C_E-C)  \\ %: mole/meter  3/second  # p_open, Pr
J_1 &= \frac{4 P_{C_E}}{r}\frac{\dvER}{\VVT} \, \frac{(C e^{- \frac{2\dvER}{\VVT}} - C_E)}{e^{-\frac{2\dvER}{\VVT}}-1} \\ %: mole/meter  3/second  # dv_ER
J_p &= \frac{2 r\, d_{ER}}{\dRtwo}  \,\frac{\Omega_u\eta_p}{N_A}   \frac{C^2}{C^2 + K_P^2} \\ %: mole/meter  3/second # d_ER, N_A, Omega_u, eta_p, K_p
\end{align}
Consider the units (brackets denotes units): 
\begin{align}
[N_A] &= \mole^{-1} \\
[P_r] &= \mu \meter \Hz \\
[P_{C_E}] &= \mu \meter \Hz \\
[\dvER] &= \text{mVolt} \\
[\VVT] &= Volt \\
[\dRtwo] &= (\mu\meter)^2 \\
[r] = [\dER] &= \mu\meter \\
[p_{open}] &= 1 \\
[K_P] &= \mu \molar \\
[\Omega_u] &= \Hz \\
[J_1] = [J_r] = [J_p] &= \Hz \, \meter^{-3} \\
[C] = [C_E] &= \mole \, \meter^{-3} 
\end{align}

\newpage
Next, consider the currents associated with $I$ (IP3) and the associated units: 
\def\kdelta{k_\delta}
\def\Kdelta{K_\delta}
\def\odelta{o_\delta}
%\def\othreeK{o_{3K}}
%\def\o5P{o_{5P}}
\def\KD{K_D}
%\def\K3{K_3}
%\def\K5P{K_{5P}}

\begin{align}
J_\beta  &= 0  \\
J_\delta &= \odelta \frac{\kdelta}{I+\kdelta} \, \frac{C^2}{C^2+\Kdelta^2} \\%: mole/second  # not sure about units, o_delta, k_delta, K_delta
J_{5P}   &= o_{5P} \, \frac{I}{I+K_{5P}}   \\ %: mole/second # o_5P, K_5P
J_{3K}   &=  o_{3K} \, \frac{C^4}{C^4+\KD^4} \, \frac{I}{I+K_3} %: mole/second # o_3K, K_D, K_3
\end{align}

and an evolution equation for $I$: 
$$
\frac{dI}{dt} = \frac{(J_\beta + J_\delta - J_{5P} - J_{3K})}{\Lambda(1-\rho)} + \text{diffusion}_I
$$

All $IP3$ currents are divided by a volume in the form $\Lambda(1-\rho)$ with units of $(\mu\meter)^3$. 

From a dimensionality point of view, 
\begin{align}
[o_{5P}] = [o_{3K}] = [\odelta] &= \mu\molar\,\Hz \\
[K_3] = [K_D] &= \mu\molar \\
[\kdelta] = [\Kdelta] &= \mu\molar \\
[\Lambda] &= (\mu\meter)^3 \\
[J_\beta] = [J_\delta] = [J_{5P}] = [J_{3K}] &= \molar \, \Hz \\
[I] &= \mole \, \meter^{-3}
\end{align}


\def\couplingCe{\text{coupling}_{CE}}
\def\electrodiffusion{\text{electrodiffusion}_{CE}}
\def\couplingelectro{\text{couplingelectro}}
\def\couplingC{\text{Coupling}_C}

\newpage
Finally, look at the various coupling terms. 
\begin{align}
[A_1] &= \frac{0.5\, F \, s}{C_m \, V_T} \, \frac{\dRtwo}{L} \\
[B_1] &= \left(1 - \frac{s \, F \, C}{2\, C_m \, V_T}\right) \frac{\dRtwo}{L}   \\
[C_1] &= \frac{\dRtwo \, C}{L}   \\
[C_0] &= \frac{-B_1 + \sqrt{B_1^2 + 4C_1 A_1})}{2 A_1} \\
[V_0] &= \frac{V_{rest} + (C_0 - C_{rest}) \, F  s}{C_m}  \\
[\couplingCe] &= \frac{4 D_{CE} }{L^2} \, (C_{E0}-C_{E1}) \\
[\electrodiffusion] &= -\frac{P_{Ca} \, V}{R \, V_T} \, \frac{C_E\, e^{-\frac{2V}{V_T}} - C} {e^{-\frac{2V}{V_T}} - 1}  \\
[\couplingelectro]  &= \frac{4 D_C}{L^2 \, V_T} \, (C_0 + C) * (V_0 - V)   \\
[\couplingC]        &= \frac{4 D_C}{L^2} \, (C_0 - C_1) 
\end{align}

Here are the parameters: 
\begin{align}
[F] &= \text{Cb} \, \mole^{-1}   \\
[C_m] &= \text{Farad} \, \meter^{-2}  = \text{Cb} \; \text{volt}^{-1} \meter^{-2} \\
[V_T] = [V_{rest}] = [V] &= \text{Volt} \\
[s] = [L] = [R] &= \mu\meter \\
[C] = [C_{rest}] = [C_E] &= \mole \; \meter^{-3} \\
[A_1] &= \text{Cb} \, \mole^{-1} \, \meter \, \text{Cb}^{-1} \, \meter^2 = \mole^{-1} \meter^3  = [C]^{-1} \\
[B_1] &= \meter \\
[C_1] &=  \meter \, [C] \\
[P_{Ca}] &= \meter^{-1} \, \Hz \\
[D_{CE}] &= \meter^2 \, \Hz 
\end{align}
\end{document}
#----------------------------------------------------------------------

\end{document}
%
