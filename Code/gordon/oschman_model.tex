\documentclass[11pt]{article}
\usepackage{amsmath}
\usepackage{amssymb}
\usepackage{amsfonts}
%\renewcommand{\mathrm}[1]{{\rm #1}}
\def\input{{\rm input}}
\def\coef{{\rm coef}}
\def\Hill{{\rm Hill}}
\def\diff{{\rm diff}}
\def\chan{{\rm chan}}
\def\pump{{\rm pump}}
\def\leak{{\rm leak}}
\def\glu{{\rm GLU}}
\def\Unif{{\rm Unif}}
\newcommand{\ca}{$\rm{Ca}^{2+}$}
\begin{document}
Description of the equations used by Evan in his dissertation, based on the Oschmann model function. 
Let us start with the equations in the Calcium. 

\subsection{Equation for Cytosolic Calcium}
$$
\frac{dc}{dt} = J_{\chan}(c,c_E,p,h) + J_{\leak}(c,c_E) - J_{\pump}(c) + J_{\diff,c}
$$
and the following definitions for the currents: 
\begin{align}
  J_{\chan} &= r_c m_\infty(p)^3 n_\infty(c)^3 h^3 (c_E-c) \\
  J_{\leak} &= r_l (c_E - c) \\
  J_{\pump} &= v_{er} \Hill(c,k_{er}, 2)
\end{align}
where we have defined the following functions: 
\begin{align}
\Hill(x, C, m) &= \frac{x^m}{x^m+C^m} \\
q_2(p) &= d_2 * \frac{p +d_1}{p + d_3}\\
m_\infty(p) &= \Hill(p, d_1, 1)  \\
n_\infty(c) &= \Hill(c, d_5, 1)  \\
h_\infty(c, p) &= \frac{q_2(p)}{q_2(p) + c}
\end{align}

\subsection{Equation for Calcium in the ER}
$$
\frac{dc_E}{dt} = -\frac{1}{\rho_A} (J_{\chan}(c,c_E,p,h) + J_{\leak}(c,c_E) - J_{\pump}(c)) + J_{\diff,c_E}
$$
where $\rho_A= ......$

\subsection{Equation for IP3 }
$$
\frac{dI}{dt} = {\rm input} + v_\delta(c,p)  - v_{3k}(c,p) - v_{5p}(p)
$$
where the currents are: 
\begin{align}
v_\delta(c,p) &=  O_\delta * (1 - \Hill(k_\delta, p, 1) ) * \Hill(c,k_{PLC\delta},2)  \\
v_{3k}(c,p) &= O_{3k} * \Hill(c,k_d,4) * \Hill(p, k_3, 1) \\
v_{5p}(p) &= O_{5p} p
\end{align}

\subsection{Equation for $h$}
$$
\frac{dh}{dt} = \frac{h_\infty(c,p)-h}{\tau_h(c,p)}
$$
where
\begin{align}
\tau_h(c,p) &= \frac{1}{a_2 (q_2(p) + c)}
\end{align}

%%% I do not like "diff" in italics. Find a way to fix this. Not good in dissertation either. 

\subsection{Diffusion Terms}
The diffusion terms are found in method $\rm{vcompartment_fe.cpp}$ and in $\rm{branch.cpp}$.  
We find that 
$$
J_\diff = v_\coef * \Delta
$$
where 
$$
v_\coef = dt * D
$$
and $D$ is a generic diffusion coefficient. 
Next, the Laplacian. 
\begin{align}
\Delta(v_i) &= \sum_b \frac{4}{(L_i + L_b)^2}  \left(\frac{R_b^2(1-\rho_{A,b})}{R_i^2(1-\rho_{A,i})} v_b - v_i\right) \\
&= \sum_b \frac{4}{(L_i + L_b)^2} * (\xi_b v_b - v_i)
\end{align}
where 
$$
\xi_b = \frac{R_b^2(1-\rho_{A,b})}{R_i^2(1-\rho_{A,i})}
$$
and $\Delta(v_i)$ is the discrete Laplace operator applied to $v_i$ ($i$ is the branch index). 

\subsection{IP3 input}
If the noise is Poisson, 
$$
\input = \bar{v}_{beta} \Hill(\glu, k_r (1+\frac{k_p}{k_r} \Hill(c, k_{pi}, 1)), .7)
$$
where 
$$
\glu = 70\; U(0,1) P(t, w_h)
$$
where $w_h$ is the half-width of the Poisson pulse, specified by the user 
and $P(t, w_h)$ is a Poisson pulse generation function. $U(0,1)$ is random variable sampled
from a Uniform distribution. 

\subsection{Morphological parameters}
\begin{verbatim}
//From branching: 

branches.back()->radius = g.R0; // soma radius

branches.back()->r_ER = sqrt( g.det_rho * pow(branches.back()->radius - .1*branches.back()->radius, 2.) );  //choosing rho_soma = .15 and radius_soma  =R0. we get r_ER = sqrt(rho_bar * ( R-.1R )^2) = 1.94

branches.back()->rho = pow(branches.back()->r_ER,2.) / pow(branches.back()->radius,2.)  ; // rho = r_ER^2 / R^2

branches.back()->rho_A = branches.back()->rho / ( 1. - branches.back()->rho )  ; // Compute rho_A from rho
\end{verbatim}

%----------------------------------------------
\subsection{List of all parameters, taken from C++ code}
\begin{table}[h]
\begin{tabular}{|lll|}
\hline
Symbol & Value & Units \\ \hline
$r_c$ $(v_{IPR})$ & 6 & 1 \\
$r_l$ $(v_l)$  & 0.11  &\\
$v_{er}$ & 4.4  &\\
$k_{er}$ & 0.05 &\\ \hline
$d_{1}$ & 0.13  &\\
$d_{2}$ & 1.049  &\\
$d_{3}$ & 0.9434 &\\
$d_{5}$ &  0.08234 &\\ \hline
$a_{2}$ &  0.2 &\\ \hline
$k_{ER}$ & 0.05 &\\
$k_{d}$ & 0.7 &\\
$k_{\delta}$ & 1. &\\
$K_{\delta}$ & 0.5  &\\
$k_{\pi}$ & .6 &\\
$k_3$ $(k_{3K})$ & 1. &\\     % or k_{3K}
$k_R$ & 1.3 &\\
$k_p$ & 10. &\\ \hline
$O_{\delta}$ & 0.025 &\\
$O_{3k}$ & 2. &\\
$O_{5p}$ & 0.05 &\\ \hline
$L_i$ & 0.4 & \\ 
$R_b$ & 0.4 & \\  
$\rho_{A,i}$ & & \\  
$D_c, D_{c_E}, D_I$ & & \\ \hline
\end{tabular}
\end{table}

%----------------------------------------------------------------------
\subsection{Table from Evan's dissertation}
\begin{table}
\centering
\begin{tabular}{ |l|l|l|l| }
\hline
Symbol & Description & Value & Units       \\
\hline
  $v_{IPR}$         & Maximal \ca release rate by IPR   &   6. & s$^{-1}$    \\
  $v_{ER}$         & Maximal \ca uptake rate by SERCA pumps  &   4.4 &  $\mu\mathrm{M}s^{-1}$    \\
  $v_l$         & \ca leak rate  &   .11   & $s^{-1}$ \\
  $O_\beta$         & Maximal rate of IP$_3$ production by PLC$\beta$  &   5.5  &   $\mu\mathrm{M}s^{-1}$  \\ \hline
  $O_\delta$    & Maximal rate of IP$_3$ production by PLC$\delta$  &   .025   &   $\mu\mathrm{M}s^{-1}$\\
  $O_{3K}$    & Maximal rate of IP$_3$ degradation by IP$_3$3K  &   2.   &  $\mu\mathrm{M}s^{-1}$ \\
  $O_{5P}$         & Maximal rate of IP$_3$ degradation by IP-5P  &   .05  & $s^{-1}$   \\
    % \hline
    % \multicolumn{4}{ |c| }{}        \\
    % \hline
  $k_{ER}$      &  Affinity for SERCA pumps &   .05   &   $\mu\mathrm{M}$  \\
  $K_{3K}$      & IP$_3$ affinity of IP$_3$3K  &   1.&     $\mu\mathrm{M}$  \\
  $K_d$      & \ca affinity of IP$_3$3K   &   .7     &   $\mu\mathrm{M}$ \\
  $\kappa_\delta$      & Inhibiting IP$_3$ affinity of PLC$\delta$  &   1. &   $\mu\mathrm{M}$   \\
  $K_\delta$      & \ca affinitiy of PLC$\delta$  &   .5 &  $\mu\mathrm{M}$    \\
  $d_1$      & IP$_3$ binding affinity   &   .13     &$\mu\mathrm{M}$  \\
  $d_2$      & Inactivating \ca binding affinity  &   1.049  &  $\mu\mathrm{M}$   \\
  $d_3$      & IP$_3$ binding affinity  &   .9434    &  $\mu\mathrm{M}$  \\
  $d_5$      & Activating \ca binding affinity  &   .08234    & $\mu\mathrm{M}$  \\
  $a_2$      & Inactivating \ca binding rate  &   .2     & $\mu\mathrm{M}s^{-1}$  \\
  \hline
\end{tabular}
\caption{\textbf{DePitta Model Parameters} Default model parameters which will be used in simulations in Chapter(chp:spatial\_experiments) unless otherwise mentioned.}
\end{table}
\end{document}


